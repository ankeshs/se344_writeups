\documentclass[pdftex,12pt,a4paper]{article}
\usepackage[pdftex]{graphicx}
\usepackage{fancyhdr}
\usepackage{geometry}
\usepackage{draftcopy}
\usepackage{float}
\usepackage{amsmath}
\usepackage{algorithm2e}
\usepackage{color, colortbl}
\definecolor{Gray}{gray}{0.9}
\renewcommand{\thesection}{\arabic{section}.}
\renewcommand{\thesubsection}{\arabic{section}.\arabic{subsection} }
\renewcommand{\headrulewidth}{0pt}
\renewcommand{\footrulewidth}{0.5pt}
\pagestyle{fancy}
\fancyhead{}
\fancyfoot[LE,LO]{\footnotesize{
SE344, Chemistry and Our Environment
}
}

\title{\vspace{-15pt}Radioactive Materials\\ SE344: Chemistry and Our Environment}
\author{Ankesh Kumar Singh (Y9090)}
\date{18th March, 2013}
\begin{document}
\maketitle
\begin{tabular}{p{370pt}}
\textbf{Keywords: }uses of radioactive materials, radioactivity regulations, transportation of radioactive materials
\end{tabular}
\vspace{10pt}\\
\hrule
\vspace{10pt}
Radioactive materials need to be handled differently from conventional chemicals as they are extremely dangerous even in very small amounts. There are strict rules and regulations for handling such chemicals, so that they are not spilled by mistake or on purpose. Nuclear energy power plants need to be taken down after a fixed lifetime, typically 40 years. After this the entire area is cleaned up.

Materials used in health care for diagnostic tests have a very short half-life. These nuclides are used to identify cancer metastases, detect heart damage or dysfunction (stress test), and to identify tumors. Nuclides also are used to treat cancer. Iodine-131 is used to treat thyroid cancer and hyperthyroidism. Strontium-89 is used to treat bone pain in advanced cancer patients. Palladium-103 and Iodine-125 implants are used to treat prostate cancer. Removable Iridium-192 and Cesium-137 implants are used to treat other types of cancers.

Radionuclides play a huge role in research. Radioactive materials are incorporated into ``labeled" chemicals. Biochemical researchers use short-lived nuclides for genetic engineering research identify biochemical and cellular changes that may otherwise not be easily seen. 

Many industrial nuclides are found in devices called ``fixed gauges". Fixed gauges contain a sealed radioactive capsule that is mounted in a fixed position on a pipe, conveyor belt, or process line. Radioactive sources move across the width of huge paper-making machines to ensure that the paper quality remains constant. Radioactive materials as sealed sources are used at construction sites to measure moisture content and density of construction materials, including asphalt paving mix.

They are also used in smoke detectors and energy saving lights. Food items and other goods are $\gamma$ irradiated to remove pests and disease carrying insects. This reduces fumigation that is both toxic and harmful to ozone layer. Electronic components of computers and mobile phones contain tantalum which is mined along with many radioactive inner transition elements. It is also used in advanced materials for aircraft engines and medical implants.

Nuclear power generation industry accounts for only 5\% of materials transported. Transport must be safe, secure, reliable and yet effective and efficient. Transportation by air takes place for materials with short half life ($\approx$30days). Most transportation takes place by road in proper containers. These containers have an inner thick lead container followed by 1 or more large boxes outside that ensure no radiations escape from it. 

BARC is the regulatory body for all radioactive materials in India.
\end{document}