\documentclass[pdftex,12pt,a4paper]{article}
\usepackage[pdftex]{graphicx}
\usepackage{fancyhdr}
\usepackage{geometry}
\usepackage{draftcopy}
\usepackage{float}
\usepackage{amsmath}
\usepackage{algorithm2e}
\usepackage{color, colortbl}
\definecolor{Gray}{gray}{0.9}
\renewcommand{\thesection}{\arabic{section}.}
\renewcommand{\thesubsection}{\arabic{section}.\arabic{subsection} }
\renewcommand{\headrulewidth}{0pt}
\renewcommand{\footrulewidth}{0.5pt}
\pagestyle{fancy}
\fancyhead{}
\fancyfoot[LE,LO]{\footnotesize{
SE344, Chemistry and Our Environment
}
}

\title{\vspace{-15pt}Sustainability\\ SE344: Chemistry and Our Environment}
\author{Ankesh Kumar Singh (Y9090)}
\date{1st April, 2013}
\begin{document}
\maketitle
\begin{tabular}{p{370pt}}
\textbf{Keywords: }population growth, sustainable society, connections in nature, natural capital
\end{tabular}
\vspace{10pt}\\
\hrule
\vspace{10pt}
Sustainability is the capacity to endure. In ecology the word describes how biological systems remain diverse and productive over time. Long-lived and healthy wetlands and forests are examples of sustainable biological systems. For humans, sustainability is the potential for long-term maintenance of well being, which has ecological, economic, political and cultural dimensions.\\

With each new scientific discovery or invention, there has been a boost to the growth of population. With increasing population, there has been an increase in urbanization to accommodate more people in smaller area. To employ all population there has been a rise in service sector. More and more agricultural land is converted into housing and corporate office complexes. As a result, food items have to be transported from longer distances, leading to rise in prices. Further, more forests are cleared to get agricultural land, leading to loss of species and disruption of ecological cycles that depend on plants. More and more wastes are generated due to industrial and urban establishments. The situation is particularly alarming in developing countries like India and China.\\

Industrialization and use of transportation means has lead to over dependence on petroleum and coal. Use of plant products as a carbon source has been proposed by means of bioethanol. Apart from fuel, a carbohydrate by enzymatic action may form a monomer that can be polymerized. However, a situation where fuel competes with food items needs to be avoided.\\

Due to uncontrolled population growth, resources are consumed, degraded and depleted. Food, hygiene and health are basic requirements for a good life. Loss of natural resources disrupts world economy by setting in a cascade of reactions. This disruption leads to crisis and war. War on Iraq and Afghanistan had much to do with rise of global oil prices thereafter.\\

A solution to this problem is to control population and resource use. However, disparities in distribution of resources often frustrate such efforts. There is also a need to spread awareness about the environment and sustainable practices. Living sustainably means living off earth's natural income without depleting or degrading the natural capital that supplies it. This natural capital consists natural resources and natural services. Resources consist of materials (both renewable like wood and non renewable like coal and petroleum) and energy (solar capital). Natural services are provided by functions of nature like purification of water and cycling of nutrients.
\end{document}