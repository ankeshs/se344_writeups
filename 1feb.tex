\documentclass[pdftex,11pt,a4paper]{article}
\usepackage[pdftex]{graphicx}
\usepackage{fancyhdr}
\usepackage{geometry}
\usepackage{draftcopy}
\usepackage{float}
\usepackage{amsmath}
\usepackage{algorithm2e}

\renewcommand{\thesection}{\arabic{section}.}
\renewcommand{\thesubsection}{\arabic{section}.\arabic{subsection} }
\renewcommand{\headrulewidth}{0pt}
\renewcommand{\footrulewidth}{0.5pt}
\pagestyle{fancy}
\fancyhead{}
\fancyfoot[LE,LO]{\footnotesize{
SE344, Chemistry and Our Environment
}
}

\title{\vspace{-15pt}Runaway Greenhouse Effect and Ozone Depletion\\ SE344: Chemistry and Our Environment}
\author{Ankesh Kumar Singh (Y9090)}
\date{7th February, 2013}
\begin{document}
\maketitle
\begin{tabular}{p{370pt}}
\textbf{Keywords: }Runaway greenhouse effect, CFCs, global warming, climate regulation, polar ice caps
\end{tabular}
\vspace{10pt}\\
\hrule
\vspace{10pt}
Biodegradation is the chemical dissolution of materials by bacteria or other biological means. Biodegradation is an important process because waste materials are decomposed by naturally occuring organisms. \textbf{Biodegradability} is not only a property or characteristic of a substance, but is also a system’s concept, i.e. a system with its conditions determines whether a substance within it is biodegraded. When material is released into the environment, its fate depends upon a whole range of physiochemical processes and its interaction with living organisms. \\

The most stable compound of carbon is carbon dioxide. All the more reduced organic compounds are thermodynamically unstable and will be randomly attacked by microbial enzymes, provided that they have some structural similarity to naturally occurring substrates. Biodegradable means that a material has the proven capability to decompose in the most common environment where the material is disposed of within 3 years through natural biological processes into nontoxic carbonaceous soil, water, carbon dioxide or methane.\\

\textbf{Primary Biodegradation} is minimal transformation that alters the physical characteristics of a compound while leaving the molecule largely intact. Partial biodegradation is not necessarily a desirable property, since the intermediary metabolites formed can be more toxic than the original substrate. Therefore, mineralization is the preferred aim.\\

\textbf{Ultimate Biodegradation} is complete biodegradation of substrate. Molecular cleavage must be sufficiently extensive to remove biological, toxicological, chemical and physical properties associated with the use of the original product, eventually forming carbon dioxide and water.

\section*{Assays for Biodegradability Test}

\end{document}